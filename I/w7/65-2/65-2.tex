\textit{This accompanying text was already checked, but is included again for consistency} \\
In general, tailing a file would be easier in \texttt{C} than it would be in \texttt{C++} as \texttt{C} is a system programming language, and \texttt{C++} is not, at least not without calling \texttt{C}-functions. This implies that there is usually one or more more level(s) of abstraction between a file and the program in \texttt{C++}. In this case, this abstraction is in the form of a buffer. When buffering, or reading, a file is blocked from being accessed by another program, possibly the one that may be adding the additional information that we are interested in to the file. Hence, more low-level access to a file (through a more low-level language) would definitely better facilitate such functionality.
