\documentclass[a4paper]{article}
\usepackage[margin=2cm]{geometry}
\usepackage{fancyhdr}
\pagestyle{fancy}
\fancyhead[CO,CE]{
                  Programming in C/C++ \\
                  Tjalling Otter \& Emiel Krol
                 }

\usepackage{etoolbox}
\usepackage{placeins}
\usepackage{float}
\usepackage{graphicx}

\usepackage[usenames,dvipsnames]{color}  %% Allow color names

\setlength\parindent{0pt}

% New page for every section
\let\stdsection\section
\renewcommand\section{\newpage\stdsection}

% New page for every subsection
%\let\stdsubsection\subsection
%\renewcommand\subsection{\newpage\stdsubsection}

% Use for source files
\usepackage{listings}
\lstdefinestyle{cc}{
  title=\lstname,
  belowcaptionskip=1\baselineskip,
  language=C++,
  breaklines=true, %% Wrap long lines
  basicstyle=\small\ttfamily,
  commentstyle=\color{Gray},
  stringstyle=\color{Black},
  keywordstyle=\bfseries\color{OliveGreen},
  identifierstyle=\color{blue},
  numbers=left,
  showstringspaces=false
}

% Use for inline code display in tex
\usepackage{listings}
\lstdefinestyle{in}{
  title=\lstname,
  belowcaptionskip=1\baselineskip,
  language=C++,
  breaklines=true, %% Wrap long lines
  basicstyle=\small\ttfamily,
  commentstyle=\color{Gray},
  stringstyle=\color{Black},
  keywordstyle=\bfseries\color{OliveGreen},
  identifierstyle=\color{blue},
  numbers=left,
  showstringspaces=false,
  aboveskip=-0.5cm
}

% Use for inline display of output/code without highlighting

\lstdefinestyle{text}{
  title=\lstname,
  belowcaptionskip=1\baselineskip,
  breaklines=true, %% Wrap long lines
  basicstyle=\small\ttfamily,
  showstringspaces=false,
  breakautoindent=true,
  numbers=left,
  showlines=true
}

%\newtoggle{aftersection}
%\preto{\lstinputlisting}{\filbreak\global\toggletrue{aftersection}}
%\preto{\subsection}{\iftoggle{aftersection}{\global\togglefalse{aftersection}}{\filbreak}}
%\newcommand{\clearpageafterfirst}{%
%  \gdef\clearpageafterfirst{\clearpage}%
%}

\begin{document}

\section*{Week 1: II}

\subsection*{Exercise 1}
\textit{Why is First::fun called?} \\
The namespace of the function that is called is determined by the argument used in the function call. In that case, this is First, and therefore First::fun() is called.

\textit{How would Second::fun be called?} \\
By explicitly calling it using its namespace: Second::fun().

\textit{How is the << operator simplified by a Koenig lookup?} \\
As the << operator from the standard library is the only defined or declared within the bounds of the program that takes the arguments as specified, there is no confusion as to what the intended use of the << operator is. Therefore, it can be used simply as is typically seen in programs, rather than its full form std::operator<<(std::cout, "string"). If there are multiple different possibilities as to the desired operator (i.e. function), then it would not work like this.

\textit{What happens if another fun(First::Enum) is defined above main()?} \\
This is an example of what was described before. The call to fun() is now ambiguous, because there are two functions that take an enum variable from the namespace First named fun. Hence, the compiler does not know which of the two to choose; it is not evident.


\newpage
\subsection*{Exercise 2}
\textit{Why doesn't this work?} \\
First of all, str is not yet defined, so it cannot be passed to promptGet. Secondly, and more to the point, the function promptGet is to return a boolean variable, but the getline function returns a pointer, not a boolean value. Up to now, we have seen getline used in such loops directly (i.e. while(getline()), but this only works because it is converted to fit a boolean comparison. \\

\textit{Change promptGet's body so that the code does compile.} \\
One option would be to alter the return function to make use of a simple if-statement:

\begin{lstlisting}[style=in]
bool promptGet(istream &in, string &str)
{
  cout << "Enter a line or ^D\n";     // ^D signals end-of-input

  if (getline(in, str))
    return true;

  return false;
}
\end{lstlisting}

\textit{Without changing promptGet's body, change promptGet so that the code does compile.} \\
As the return type of getline is \&istream (in this case), we can change the header of promtGet to accord, as follows:

\begin{lstlisting}[style=in]
istream &promptGet(istream &in, string &str)
{
  cout << "Enter a line or ^D\n";     // ^D signals end-of-input

  return (getline(in, str));
}
\end{lstlisting}

Now, the implicit operators as mentioned before are used in main to allow for their use in the while-statement.


\newpage
\subsection*{Exercise 3}
\lstinputlisting[style=cc]{../3/main.cc}
\lstinputlisting[style=cc]{../3/strings/strings.ih}
\lstinputlisting[style=cc]{../3/strings/stringshandin.h}
\lstinputlisting[style=cc]{../3/strings/operatorIndex.cc}
\lstinputlisting[style=cc]{../3/strings/indexoperator1.cc}
\lstinputlisting[style=cc]{../3/strings/indexoperator2.cc}

\newpage
\subsection*{Exercise 4}
\lstinputlisting[style=cc]{../4/main.cc}
\lstinputlisting[style=cc]{../4/strings/strings.h}
\lstinputlisting[style=cc]{../4/strings/strings.ih}
\lstinputlisting[style=cc]{../4/strings/extractFrom.cc}
\lstinputlisting[style=cc]{../4/strings/insertInto.cc}

\newpage
.
\newpage
\subsection*{Exercise 6}
\lstinputlisting[style=cc]{../6/msg.h}
\lstinputlisting[style=cc]{../6/main.ih}
\lstinputlisting[style=cc]{../6/main.cc}
\lstinputlisting[style=cc]{../6/show.cc}
\newpage
Output of the given program:
\begin{lstlisting}[style=text]
NONE
EMERG




\end{lstlisting}


\end{document}
