\begin{itemize}
  \item Draw \texttt{Multi}'s class hierarchy
\end{itemize}
Below is the class hierarchy of \texttt{Multi} at this point of the assignment. \\

\tikzstyle{every node}=[draw=black,thick]
\tikzset{sibling distance=18pt}
\begin{tikzpicture}[grow=up]
\Tree [.Multi [.Deriv1 Basic ] [.Deriv2 Basic ] ]
\end{tikzpicture}

\begin{itemize}
  \item Explain the compiler's error message after the addition of the \texttt{static\_cast}
\end{itemize}

As can be seen from the illustration above, due to the way that \texttt{Deriv1} and \texttt{Deriv2} are constructed, \texttt{Basic} is included twice by the time \texttt{Multi} inherits from \texttt{Deriv1} and \texttt{Deriv2}. Hence, the compiler indicates that it does not know which \texttt{Basic} to cast to.

\begin{itemize}
  \item Change the statement so that there is no compilation error
\end{itemize}
First, the cast can be done in a two-step fashion: first to either \texttt{Deriv1} or \texttt{Deriv2}, and then to its associated \texttt{Basic}. By doing so, the compiler knows which \texttt{Basic} parent is applicable. This is achieved as follows: \\

\begin{lstlisting}[style=in, caption=c\_multi.cc]
Multi::Multi()
{
  cout << static_cast<Basic *>(static_cast<Deriv1 *>(this)) << '\n';
}
\end{lstlisting}

Secondly, a \texttt{reinterpret\_cast} can be used instead, as follows. What this does is blindly interprets the \texttt{Multi} pointer (\texttt{this}) as a \texttt{Basic} pointer. Note that this is a very dangerous practice, and should be used with extreme caution, as there is a myriad of problems that can arise from this. \\

\begin{lstlisting}[style=in, caption=c\_multi.cc]
Multi::Multi()
{
  cout << reinterpret_cast<Basic *>(this) << '\n';
}
\end{lstlisting}

\begin{itemize}
  \item Show the required modifications to allow the compiler to compile the statement without errors
\end{itemize}

The best way to solve the compilation error without altering the statement would be to make use of virtual inheritance. As such, the class interface of \texttt{Deriv1} and \texttt{Deriv2} should be changed as follows: \\

\begin{lstlisting}[style=in, caption=deriv1.h]
...
class Deriv1: public virtual Basic
{
};
...
\end{lstlisting}
\begin{lstlisting}[style=in, caption=deriv2.h]
...
class Deriv2: public virtual Basic
{
};
...
\end{lstlisting}

\begin{itemize}
  \item How do you realize that this 2nd constructor is the only Basic constructor that's called
\end{itemize}
Without specifying otherwise, \texttt{Multi} directly calls the default constructor of \texttt{Basic}. To change this behaviour, explicitly calling the integer constructor in the initialisation list of \texttt{Multi}, as shown below, is the correct approach.
\begin{lstlisting}[style=in, caption=c\_multi.cc]
#include "multi.ih"

Multi::Multi()
:
  Basic(10)
{
  cout << static_cast<Basic *>(this) << '\n';
}
\end{lstlisting}
