Here, I used the following approach to shed the excess capacity of the \texttt{vector} in the class. Using the \texttt{swap} function, the data member \texttt{d\_vWords} is replaced with an anonymous new \texttt{vector} constructed directly using the original \texttt{d\_vWords}. In this process, the size and capacity of the anonymous (and new \texttt{d\_vWords}) are immediately set appropriately.
\texttt{shrink\_to\_fit} should not be used because, as stated, it is merely a request to the compiler to shed capacity. It is therefore not always executed, even though it seems to constitute an explicit command. Furthermore, in a class environment, it makes more sense to incorporate a full 'clean up' of (all) its data allocation, as it were, and to ensure that these instructions are actually executed.
