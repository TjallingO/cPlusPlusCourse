Consider the example program below. It is a bit contrived, because it is clear that this would lead to memory leaks. However, if one imagines this type of condition to be part of a much, much larger program, then it can serve as an illustration of why using \texttt{exit()} is a bad idea. Say that the string allocation is part of the construction of a certain class. Additionally, its destructor would return this memory, represented by the last line in this example. However, along the way, something horrible happens which should prompt the program to stop. This way of achieving that goal would mean that the destructor is not called, thus causing a memory leak. Furthermore, unless properly embedded in a function that is more descriptive, does not provide any form of information about what went wrong, or why. Throwing an exception in this case would allow for the programming of a deallocation procedure for when such a situation occurs, but more importantly, forces the programmer to think about the proper approach that may even allow the program to continue functioning despite this error. If not, a try/catch block would allow the program to come to a more graceful ending, whereby at least destructors are called.
