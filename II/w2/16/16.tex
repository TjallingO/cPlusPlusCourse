Consider the example program below. The program is as simple as it can be, it just creates one local variable. However, as it needs to store a rather long string, the object allocates more memory of its own. If the string were shorter, the exit call would not lead to problems, but now it does. Namely, the destructor of the class string is not called, and its own allocated memory is not freed. Therefore, this is not a proper way to exit a program. It is also not very descriptive, as the user has no idea what went wrong. Not only can an exception show the user a string that explains what prompted the program to end, but it also constitutes a normal end of the program, which would call destructors as per usual. Lastly, implementing an exception handler forces the programmer to think about the logical flow of a program, and what should actually happen if an exceptional situation arises.
