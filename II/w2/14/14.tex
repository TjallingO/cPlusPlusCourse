As can be seen, there are two try blocks in this program. The first is the one in \texttt{main()}, which tries to construct an array of ten \texttt{MaxFour} objects, and catches any string-type exception, printing its message if encountered. This is just for the benefit of the user. The second is in the default constructor of the class. The constructor first allocates a new string in \texttt{d\_content}, increments the object counter, and throws an exception if this counter surpasses four objects. In that case, the current object is incomplete, and a destructor would not be called for it at the program's end. This would result in a memory leak with the size of a single object, because after this incomplete objects the array construction is halted. However, a simple \texttt{catch(...)} block that follows this try block with the same content as the destructor is enough to delete what has been allocated, effectively constituting object destruction right there and then.

The reason this is so simple is because this manner of programming is very intuitive. The constructor is asked to perform allocation, and if something goes wrong, to revert that allocation.
