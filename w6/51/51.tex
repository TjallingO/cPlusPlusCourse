\subsubsection*{New / delete variants}

\textit{New variable or array} \\
Allocates new memory sized appropriately to the type and dimensions specified. Thereafter it will attempt to construct and initialise these objects. Finally, it will return a pointer to the start of the memory allocated to these objects. The advantage and consideration of constructing an object (array) is that one has control over when it is deleted (if at all...).

\begin{lstlisting}[style=inline]
int main()
{
  Class *pointer;
  if (boolAppropriate)
    pointer = new Class;
    // or
    pointer = new Class[5];

  if (boolNoLongerNecessary)
    delete pointer;
    // or
    delete[] pointer;
}
\end{lstlisting}

Without the use of new (i.e. Class newClass), the scope of the newly constructed class would simply be limited to within the if statement, and thereafter destroyed. However, now the object persists until manually deleted. As such, the creation and destruction do not take place unless necessary. Another example would be in the creation of arrays with dimensions that are not previously established, and re-pointing a pointer to the new array. \\
\\
\textit{Operator and placement new} \\
Allows for the allocation of raw memory that can be cast to fit a certain amount of a specific type. It is useful if one already has a reasonable idea of how much memory is needed for a specific application. Thereafter, a pointer to these memory locations is returned.

\begin{lstlisting}[style=inline]
#include <memory>

void destroy()
{
  for (string *end = d_storage + d_size; end-- != d_memory)
    end->~string();
  operator delete(d_memory);
}

int main()
{
  string *tmp = static_cast<string *>(operator new(d_capacity * sizeof(string *)));
  // Raw memory is allocated, tmp points there

  for(size_t idx = d_size; idx--; )
    new(tmp + idx) string{ d_storage[idx] }
  // Old strings copied into newly allocated memory

  destroy();
  // Old memory deleted
  d_memory = tmp;
  // d_memory now points to new memory, which contains the initial strings
  new (d_memory + d_size) std::string{ newString };
  ++d_size;
  // Placement new adds a new string to the memory
  // ...
  destroy(); // one last time
}
\end{lstlisting}

In this example, we know that a block of memory sized to fit $d\_capacity$ strings is required, but not yet which constructor will be used to fill it (which comes later). Hence, it can already be allocated, but left empty until that decision is made (or even if it is not made at all).
