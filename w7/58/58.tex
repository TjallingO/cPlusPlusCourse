\begin{table}[H]
\parbox{0.50\linewidth}
{
\centering
\begin{tabular}{l|c}
\textit{\textbf{Type}} & \textit{\textbf{Time}} \\ \hline
\textbf{real}          & 0m2,944s             \\
\textbf{user}          & 0m2,924s             \\
\textbf{sys}           & 0m0,020s              \\ \hline
\end{tabular}
\caption{Not prefixed}
}
\parbox{0.5\linewidth}
{
\centering
\begin{tabular}{l|c}
\textit{\textbf{Type}} & \textit{\textbf{Time}} \\ \hline
\textbf{real}          & 0m0,033s               \\
\textbf{user}          & 0m0,033s               \\
\textbf{sys}           & 0m0,000s               \\ \hline
\end{tabular}
\caption{If-prefixed}
}
\parbox{0.5\linewidth}

\end{table}


without the if check in every iteration the string has to be stored in the
buffer. Afterwards when it tries to pass it to out, it wont be printed since
the state is set to failbit.
With the if check there is no need to store things in the buffer, which makes
it much faster.
So a general rule should be to perform checks before storing things in a buffer
when possible.
