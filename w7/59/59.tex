The code does work as intended for the first operation as the default open mode of \texttt{ofstream} is \texttt{ios\_base::out}. This means that it immediately tries to write to the file, and create said file if it is not available. Conversely, \texttt{fstream} is meant for both reading and writing, and thus its default open mode is \texttt{ios\_base::in | ios\_base::out}: first open the file, then write to it. Since it does not exist, it can't open it, and won't continue.

The code can be fixed in a number of ways. First, the second operation could simply make use of \texttt{ofstream} as well. In this case, this is probably the best solution as the use of \texttt{ofstream} indicates that this concerns a 'write-only' application. Secondly, the default open mode of \texttt{fstream} can be overridden using the following parameter \texttt{out2\{ "./tmp/out2", ios\_base::out \};}. Third, the file could be created first, before running the operation; either manually or by some other means in the code itself.
